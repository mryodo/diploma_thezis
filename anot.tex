\documentclass[hyperpdf, nobind, oneside, 16pt]{hepthesis}
%% For Cambridge hard-bound version (must be one-sided)
%\documentclass[hyperpdf,oneside]{hepthesis}

\usepackage[T2A]{fontenc}
\usepackage[utf8]{inputenc}
\usepackage[russian]{babel}
\usepackage{amsfonts}
\usepackage{amsmath}
\usepackage{graphicx}
%\graphicspath{/images}
\usepackage{float}
\usepackage{caption}
\usepackage[list=true,listformat=simple]{subcaption}

\linespread{1.5}

%\usepackage{subfigure}
%\setcounter{lofdepth}{3}  %\replace depth number with an integer number

%\usepackage[sorting=none]{biblatex}
\def\bbljan{Jan.}
\def\bblfeb{Feb.}
\def\bblmar{Mar.}
\def\bblapr{Apr.}
\def\bblmay{May}
\def\bbljun{Jun.}
\def\bbljul{Jul.}
\def\bblaug{Aug.}
\def\bblsep{Sep.}
\def\bbloct{Oct.}
\def\bblnov{Nov.}
\def\bbldec{Dec.}

\usepackage{titlesec}
\titleformat{\chapter}{\normalfont\huge\bfseries}{\thechapter.}{20pt}{\huge\bfseries}
%% Load special font packages here if you wish
%\usepackage{lmodern}
%\usepackage{mathpazo}
%\usepackage{euler}
\usepackage{iwona}

\usepackage[resetlabels,labeled]{multibib}
\newcites{S}{Авторские публикации по теме работы}

\usepackage{tocloft}
\setlength{\cftbeforechapskip}{0.8em}

\allowdisplaybreaks
%\setacknowledgementstitle{Благодарности}

\title{
\textcyrillic{Механизмы сосуществования стационарных биологических сообществ в пространствах разных размерностей}}
\author{Антон Сергеевич Савостьянов}


\begin{document}

\thispagestyle{empty}
\section*{Аннотация}

В работе рассматривается двухвидовая  модель самоструктурирующихся стационарных биологических сообществ, предложенная У. Дикманом и Р. Лоу, учитывающая пространственную неоднородность сообщества. Разработан численный метод для изучения системы интегро-дифференциальных уравнений, описывающей положение равновесия в модели; предложены математические эвристики для численного метода, использущие преобразование Ханкеля в двумерном случае и задачу Лапласа и присоединенные полиномы Лежандра в трехмерном случае, позволяющие существенное ускорить метод; найдены нетривиальные стационарные точки; исследованы ограничения на пространство параметров модели, приводящие к подобным стационарным точкам. Полученный метод применен к ряду общеизвестных биологических сценариев.


\newpage

\thispagestyle{empty}
\section*{Annotation}

This paper studies two-species model of self-structuring stationary biological communities, proposed by U. Dieckmann and R. Law, applicable for spatially heterogeneous populations. The numerical method applicable for solving the system of integro-differential equations that describes an equilibrium state of the system is developed; number of mathematical heuristics based on Hankel transform in two-dimensional case and Laplace problem with associated Legendre polynomials are proposed; nontrivial stationary points are found; boundaries on the model's parameter space leading to aforementioned nontrivial points are examined. Developed method applied for a number of widely-known biological scenarios.

\end{document}
