\chapter{Численный метод в пространствах разной размерностей }

\section{Численный метод для решения системы}

Перепишем полученную систему в операторном виде для дальнейшей работы:
\begin{equation*}
\begin{cases}
	N_{1}=L_{1}(y_{11},y_{12},y_{21},y_{22})\\
	N_{2}=L_{2}(y_{11},y_{12},y_{21},y_{22})\\
	D_{11}=K_{11}(N_{1},N_{2},D_{11},D_{12})\\
	D_{12}=K_{12}(N_{1},N_{2},D_{11},D_{12},D_{22})\\
	D_{22}=K_{22}(N_{1},N_{2},D_{21},D_{22})
\end{cases}
\end{equation*}
где $L_i$ --- линейный интегральные операторы, включающие $ \int_{\mathbb{R}^n}w_{ij}(\xi)C_{ij}(\xi)d\xi$, а $K_{ij}$ --- интеральные операторы с нелинейностями вида $ \int_{\mathbb{R}^n}w_{ik}(\xi)D_{ik}(\xi)d\xi$, $D_{ik}(\xi)\cdot [w_{ik}*D_{ik}](\xi)$ и $[(w_{ik}\cdot D_{ik})*D_{ik}](\xi)$. 

Дискретизация полученных уравнений приведет к системе алгебраических уравнений второго порядка; для решения этой системы применим \textit{метод рядов Неймана (метод последовательных приближений)}, который сойдется при условии сжатия по норме во всех итеративных точках. 

Опишем принцип работы данного метода. Пусть дано уравнение вида \begin{equation*}
\vec{f}=\vec{g}+K\vec{f},
\end{equation*}
где $K$ --- некоторый оператор, $\vec{g}$ --- известная функция, а $\vec{f}$ --- искомая.

\textbf{Теорема 1.} Ряд 
\begin{equation*}
	\vec{g}+K\vec{g}+K^2\vec{g}+\ldots+K^n\vec{g}+\ldots
\end{equation*}
	в случае своей сходимости сходится к решению уравнения выше.\\
\textit{Доказательство. }\ Действительно, $\vec{g}+K\vec{f}=\vec{g}+K(\vec{g}+K\vec{g}+K^2\vec{g}+\ldots+K^n\vec{g}+\ldots)=\vec{g}+K\vec{g}+K^2\vec{g}+\ldots+K^n\vec{g}+\ldots=\vec{f}$, что и требовалось доказать.

\textbf{Теорема 2.} Если оператор $K$ --- сжимающий, т.е.
\begin{equation*}
	\exists \alpha,\;0<\alpha<1:\;\forall \vec{x}, \vec{y} \Rightarrow \rho(K\vec{x}, K\vec{y})<\alpha\cdot\rho(\vec{x}, \vec{y}),
\end{equation*}
	то ряд  $\vec{g}+K\vec{g}+K^2\vec{g}+\ldots+K^n\vec{g}+\ldots$ сходится.\\
\textit{Доказательство.} По теореме Банаха у сжимающего отображения в полном метрическом пространстве существует неподвижная точка, при том только одна. Значит, ряд сходится.

Далее приведем алгоритм работы численного метода:

\begin{enumerate}
	\item Инициализируем процесс $N_1=N_2=0$, $D_{11}=D_{12}=D_{22}=0$; формально говоря, в большинстве запусков в рамках данной работы инициализация будет проводится не нулевыми значениями, а известной стационарной точкой для близких параметров системы;
	\item Пересчитаем последовательно $D_{12}=K_{12}(N_1, N_2, D_{11}, D_{12}, D_{21})$;
	\item Зная новый $D_{12}$, пересчитаем:
	\begin{align*}
	N_1=L_1(y_{11}, y_{12}, y_{21}, y_{22}) \\
	N_2=L_2(y_{11}, y_{12}, y_{21}, y_{22})
	\end{align*}
	\item Пересчитаем
	\begin{align*}
	D_{11}=K_{11}(N_{1},N_{2},D_{11},D_{12})\\
	D_{22}=K_{22}(N_{1},N_{2},D_{21},D_{22})
	\end{align*}
	\item Итеративно повторим, пока суммарная относительная ошибка всех неизвестных величин не станет меньше выбранной величины; для дальнейших вычислений выбиралась метрика $\mathbb{L}_1$ для сравнения функциональных погрешностей и $\varepsilon=10^{-8}$.
\end{enumerate}

\section{Эвристики численного метода в случаях $ \mathbb{R}^2 $}

Далее обратимся к изучению двумерного случая. Наиболее сложным с вычислительной точки зрения вопросом является работа с двумерной сверткой. Для нее верно свойство преобразования Фурье: 
 \begin{equation*}
[f*g]_{\mathbb{R}^{2}}=\hat{F}[F[f]\cdot F[g]],
\end{equation*}
где $\hat{F}[h]$ --- обратное преобразование Фурье. После дискретизации на квадратной сетке размера $ K\times K $ прямое и обратное преобразования выглядят следующим образом:
\begin{equation*}
 G_{uv}=\frac{1}{K^{2}}\sum\limits _{n=1}^{K-1}\sum\limits _{m=1}^{K-1}x_{mn}e^{-\frac{2i\pi}{K}(mu+nv)},
\end{equation*}
\begin{equation*}
 x_{nm}=\sum\limits _{n=1}^{K-1}\sum\limits _{m=1}^{K-1}G_{uv}e^{\frac{2i\pi}{K}(mu+nv)}.
\end{equation*}
 Эти преобразования могут быть ускорены за счет применения одномерного быстрого преобразования Фурье: 
\begin{equation*}
G_{uv}=\frac{1}{K}\sum\limits _{n=1}^{K-1}\left[\frac{1}{K}\sum\limits _{m=1}^{K-1}x_{mn}e^{-\frac{2i\pi nv}{K}}\right]e^{-\frac{2i\pi mu}{K}}, 
\end{equation*}
что дает алгоритмическую сложность $ O(K^{3}\log(K)) $, поскольку необходимо вычислить быстрое преобразование Фурье ($ K\log(K) $) в каждом узле сетки \cite{Press1996}. Данную асимптотику можно улучшить, применив преобразование Ханкеля \cite{Baddour:09}. 

Перейдем в преобразовании Фурье к полярным координатам: 
\begin{equation*}
 F[f](\omega_{x},\omega_{y})=\int\limits _{-\infty}^{+\infty}\int\limits _{-\infty}^{+\infty}f(x,y)e^{-i(\omega_{x}x+\omega_{y}y)}dxdxy=\int\limits _{0}^{+\infty}\int\limits _{-\pi}^{\pi}f(r,\theta)e^{-ir\rho\cos(\psi-\theta)}rdrd\theta.
\end{equation*}
Пользуясь тем, что исследуемые функции радиально-симметричны, получаем соотношение 
\begin{equation*}
F[f](\rho,\psi)=\int\limits _{0}^{+\infty}rf(r)dr\int\limits _{-\pi}^{\pi}e^{-ir\rho\cos(\psi-\theta)}d\theta=2\pi\int\limits _{0}^{+\infty}rf(r)J_{0}(r\rho)dr, 
\end{equation*}
которое известно как \textbf{{преобразование Ханкеля 0-го порядка}}, где $ J_{0}(x)=\frac{1}{2\pi}\int\limits _{-\pi}^{\pi}e^{-ix\cos\tau}d\tau $ --- \textit{функция Бесселя нулевого порядка}. Если же теперь дополнительно сделать экспоненциальную замену переменных $ r=r_{0}e^{x} $, $ \rho=\rho_{0}e^{y} $, 
\begin{equation*}
H_{0}[f](\rho)=\int_{0}^{+\infty}rf(r)J_{o}(\rho r)dr=\frac{1}{e^{y/4}}[(f(e^{x/4})\cdot e^{x/4})*J_{0}(e^{x/4})](y)|_{x=\ln r,\;y=\ln\rho}.
\end{equation*}
то вместо преобразования Фурье получим свертку двух функций, алгоритмическая сложность которой $ O(K\cdot\log(K)) $, т.е. было произведено ускорение в $ K^{2} $ раз.

\section{Эвристики численного метода в случаях $ \mathbb{R}^3 $}

Аналогичным образом проведем рассуждение в случае трехмерной свертки. Для нее верно свойство преобразования Фурье: 
\begin{equation*}
[f*g]_{\mathbb{R}^{3}}=\hat{F}[F[f]\cdot F[g]],
\end{equation*}
где $\hat{F}[h]$ --- обратное преобразование Фурье. Пользуясь соображениями выше, легко понять, что ее вычислительная сложность есть не что иное, как быстрое преобразование Фурье, запущенное в каждой точке пространства, т.е. $ O(K^{3}\cdot K\log K)=O(K\cdot\log K) $. Улучшим вычислительную сложность нашего метода за счет перехода в класс радиально симметричных функций, как и ранее. Для этого сделаем дополнительное построение: вспомним задачу Лапласса с граничным условием на шаре:
\begin{equation*}
\begin{cases}
	\Delta u=0\\
	0<\rho<a,\;0<\theta<\pi,\;0<\phi<2\pi\\
	u(a,\theta,\phi)=f_{0}(\theta,\phi)\equiv0
\end{cases}
\end{equation*}
Строго говоря, мы необязаны привязывать наши рассуждения к конкретному виду начальных условий, поэтому для удобства положим, что задача дана с условиями Дирихле, как приведено выше. Перепишем:
\begin{equation*}
\begin{cases}
\frac{1}{r^2} \frac{\partial}{\partial r}\left(r^2 \frac{\partial u}{\partial r}\right) + \frac{1}{r^2 \sin \theta} \frac{\partial}{\partial \theta }\left(\sin \theta \frac{\partial u}{\partial \theta }\right)+\frac{1}{r^2 \sin^2 \theta } \frac{\partial^2 u}{\partial \phi^2}=0\\
0<\rho<a,\;0<\theta<\pi,\;0<\phi<2\pi\\
u(a,\theta,\phi)=f_{0}(\theta,\phi)
\end{cases}
\end{equation*}
Положим $u=R(r)Y(\theta, \phi)$. Тогда:
\begin{equation*}
Y \frac{d}{dr}(r^2 R')+R\left[ \frac{1}{ \sin \theta} \frac{\partial}{\partial \theta }\left(\sin \theta \frac{\partial Y}{\partial \theta }\right)+\frac{1}{\sin^2 \theta } \frac{\partial^2 Y}{\partial \phi^2}\right]=0
\end{equation*}
Откуда получаем по классической схеме постоянной разделения:
\begin{align*}
r^2R''+2rR'-\lambda R =0 \\
\frac{1}{ \sin \theta} \frac{\partial}{\partial \theta }\left(\sin \theta \frac{\partial Y}{\partial \theta }\right)+\frac{1}{\sin^2 \theta } \frac{\partial^2 Y}{\partial \phi^2}+\lambda Y = 0
\end{align*}
Так же, как и выше, для второго уравнения разделим переменные $Y(\theta, \phi)=F(\theta)G(\phi)$:
\begin{equation*}
G\frac{1}{ \sin \theta} \frac{\partial}{\partial \theta }\left( \theta F'\right)+\frac{1}{\sin^2 \theta }FG''+\lambda FG = 0
\end{equation*}
Откуда аналогичным разделением переменных получаем:
\begin{align*}
G''+\mu G = 0 \\
\frac{1}{ \sin \theta} \frac{\partial}{\partial \theta }\left( \theta F'\right)+\left(\lambda - \frac{\mu}{\sin^2\theta}\right)F=0
\end{align*}
Из первого уравнения, зная собственные числа оператора второй производной, получаем, что $\mu=n^2$. Во втором уравнении сделаем, замену $t=\cos \theta$:
\begin{align*}
G'=G'_t(-\sin\theta) \\
G''=G''_{tt}\sin^2\theta -G'_t\cos\theta
\end{align*}
Тогда мы получаем классическое уравнение:
\begin{equation*}
(1-t^2)G''_{tt}-2tG'_t+\left(\lambda - \frac{n^2}{1-t^2}\right)G=0
\end{equation*}
где собственный базис есть $ P_{n}^{k}(t) $ --- \textit{присоединенные полиномы Лежандра}, известные своей ортогональностью в $ L_{2} $: \begin{equation*}
P_{n}^{k}(t)=\frac{(-1)^{n}}{n!2^{n}}(1-t^{2})^{k/2}\frac{d^{n+k}}{dt^{n+k}}[(1-t^{2})^{n}],
\end{equation*}
т.е. 
\begin{equation*}
G(t)=\sum\limits_0^\infty A_n \frac{(-1)^{n}}{n!2^{n}}(1-t^{2})^{k/2}\frac{d^{n+k}}{dt^{n+k}}[(1-t^{2})^{n}]
\end{equation*}
Возвращаясь к неизвестной функции и меняя порядок суммирования, чтобы привести к виду метода Фурье $ u(\rho,\theta,\phi)=\sum\limits _{n=0}^{+\infty}\sum\limits _{k=-n}^{n}R_{n}^{k}(\rho)Y_{n}^{k}(\theta,\phi) $ для решения дифференциальных уравнений в частных производных, получаем: \begin{equation*}
Y_{n}^{k}(\theta,\phi)=\sqrt{\frac{(2n+1)(n-k)!}{4\pi(n+k)!}}P_{n}^{k}(\cos\phi)e^{ik\theta},
\end{equation*}
Воспользуемся разложением ядра Фурье через $ Y_{n}^{k} $ \cite{Baddour:10}: \begin{equation*}
e^{i(\vec{w},\vec{r})}=4\pi\sum\limits _{n=0}^{+\infty}\sum\limits _{k=-n}^{n}(-1)^{n}\sqrt{\frac{\pi}{2r\rho}}J_{n+1/2}(\rho r)\overline{Y_{n}^{k}(\phi,\psi)}Y_{n}^{k}(\theta,\eta)
\end{equation*}
Подставляя полученное выражение в теорему о свертке:
\begin{multline*}
[f***g]=F^{-1}[ F[f]\cdot F[g] ]= \int\limits_{\mathbb{R}^3} \left[ \left( \int\limits_{\mathbb{R}^3} f(r) e^{i (\vec{w}, \vec{r})} r^2 \sin \phi dr d\phi \right) \cdot \right. \\
\left. \cdot \left( \int\limits_{\mathbb{R}^3} g(r) e^{i (\vec{w}, \vec{r})} r^2 \sin \phi dr d\phi \right) \right] e^{-i (\vec{w}, \vec{r})} \rho^2 \sin \theta d\rho d\theta d\eta
\end{multline*}
и меняя порядок суммирования с учетом ортогональности полиномов Лежандра, получаем ненулевым только первое слагаемое с обычной полярной заменой координат: 
\begin{multline*}
[f***g]=F^{-1}[ F[f]\cdot F[g] ]=\\
= \int\limits_{\mathbb{R}^3} \left[  \int\limits_{\mathbb{R}^3 \times \mathbb{R}^3} f(r) 4\pi\sum\limits _{n=0}^{+\infty}\sum\limits _{k=-n}^{n}(-1)^{n}\sqrt{\frac{\pi}{2r\rho}}J_{n+1/2}(\rho r)\overline{Y_{n}^{k}(\phi,\psi)}Y_{n}^{k}(\theta,\eta) r^2 \sin \phi dr d\phi \right. \cdot \\ \cdot \left. g(r) 4\pi\sum\limits _{n=0}^{+\infty}\sum\limits _{k=-n}^{n}(-1)^{n}\sqrt{\frac{\pi}{2r\rho}}J_{n+1/2}(\rho r)\overline{Y_{n}^{k}(\phi,\psi)}Y_{n}^{k}(\theta,\eta)e^{i (\vec{w}, \vec{r})} r^2 \sin \phi dr d\phi \right] e^{-i (\vec{w}, \vec{r})} \rho^2 \sin \theta d\rho d\theta d\eta
\end{multline*}
Что в итоге сворачивается к следующей формуле, позволяющей также, как и в двумеоном случае, свести вычислительную сложность к сложности одномерной задачи:
\begin{equation*}
[f***g](\vec{r})=4\pi[(r\cdot f)*g](r)
\end{equation*}