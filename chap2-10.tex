\chapter{Эвристики численного метода в случаях $ \mathbb{R}^2 $ and $ \mathbb{R}^3 $ }

\section{Ускорение вычислений в двумерном случае}

Далее обратимся к изучению двумерного случая. Наиболее сложным с вычислительной точки зрения вопросом является работа с двумерной сверткой. Для нее верно свойство преобразования Фурье: 
 \begin{equation*}
[f*g]_{\mathbb{R}^{2}}=\hat{F}[F[f]\cdot F[g]]. 
\end{equation*}
После дискретизации на квадратной сетке размера $ K\times K $ прямое и обратное преобразования выглядят следующим образом:
\begin{equation*}
 G_{uv}=\frac{1}{K^{2}}\sum\limits _{n=1}^{K-1}\sum\limits _{m=1}^{K-1}x_{mn}e^{-\frac{2i\pi}{K}(mu+nv)},
\end{equation*}
\begin{equation*}
 x_{nm}=\sum\limits _{n=1}^{K-1}\sum\limits _{m=1}^{K-1}G_{uv}e^{\frac{2i\pi}{K}(mu+nv)}.
\end{equation*}
 Эти преобразования могут быть ускорены за счет применения одномерного быстрого преобразования Фурье: 
\begin{equation*}
G_{uv}=\frac{1}{K}\sum\limits _{n=1}^{K-1}\left[\frac{1}{K}\sum\limits _{m=1}^{K-1}x_{mn}e^{-\frac{2i\pi nv}{K}}\right]e^{-\frac{2i\pi mu}{K}}, 
\end{equation*}
что дает алгоритмическую сложность $ O(K^{3}\log(K)) $, поскольку необходимо вычислить быстрое преобразование Фурье ($ K\log(K) $) в каждом узле сетки. Данную асимптотику можно улучшить, применив преобразование Ханкеля CITE. 

Перейдем в преобразовании Фурье к полярным координатам: 
\begin{equation*}
 F[f](\omega_{x},\omega_{y})=\int\limits _{-\infty}^{+\infty}\int\limits _{-\infty}^{+\infty}f(x,y)e^{-i(\omega_{x}x+\omega_{y}y)}dxdxy=\int\limits _{0}^{+\infty}\int\limits _{-\pi}^{\pi}f(r,\theta)e^{-ir\rho\cos(\psi-\theta)}rdrd\theta.
\end{equation*}
Пользуясь тем, что исследуемые функции радиально-симметричны, получаем соотношение 
\begin{equation*}
F[f](\rho,\psi)=\int\limits _{0}^{+\infty}rf(r)dr\int\limits _{-\pi}^{\pi}e^{-ir\rho\cos(\psi-\theta)}d\theta=2\pi\int\limits _{0}^{+\infty}rf(r)J_{0}(r\rho)dr, 
\end{equation*}
которое известно как \textbf{{преобразование Ханкеля 0-го порядка}}, где $ J_{0}(x)=\frac{1}{2\pi}\int\limits _{-\pi}^{\pi}e^{-ix\cos\tau}d\tau $ --- \textit{функция Бесселя нулевого порядка}. Если же теперь дополнительно сделать экспоненциальную замену переменных $ r=r_{0}e^{x} $, $ \rho=\rho_{0}e^{y} $, 
\begin{equation*}
H_{0}[f](\rho)=\int_{0}^{+\infty}rf(r)J_{o}(\rho r)dr=\frac{1}{e^{y/4}}[(f(e^{x/4})\cdot e^{x/4})*J_{0}(e^{x/4})](y)|_{x=\ln r,\;y=\ln\rho}.
\end{equation*}
то вместо преобразования Фурье получим свертку двух функций, алгоритмическая сложность которой $ O(K\cdot\log(K)) $, т.е. было произведено ускорение в $ K^{2} $ раз.

\section{Ускорение вычислений в трехмерном случае}

Аналогичным образом проведем рассуждение в случае трехмерной свертки. Для нее верно свойство преобразования Фурье: 
\begin{equation*}
[f*g]_{\mathbb{R}^{2}}=\hat{F}[F[f]\cdot F[g]].
\end{equation*}
Пользуясь соображениями выше, легко понять, что ее вычислительная сложность есть не что иное, как быстрое преобразование Фурье, запущенное в каждой точке пространства, т.е. $ O(K^{3}\cdot K\log K)=O(K\cdot\log K) $. Улучшим вычислительную сложность нашего метода за счет перехода в класс радиально симметричных функций, как и ранее. Для этого сделаем дополнительное построение: вспомним задачу Лапласса с граничным условием на шаре:
\begin{equation*}
\begin{cases}
	\Delta u=0\\
	0<\rho<a,\;0<\theta<\pi,\;0<\phi<2\pi\\
	u(a,\theta,\phi)=f_{0}(\theta,\phi)\equiv0
\end{cases}
\end{equation*}
Строго говоря, мы необязаны привязывать наши рассуждения к конкретному виду начальных условий, поэтому для удобства положим, что задача дана с условиями Дирихле, как приведено выше. Полагая по методу Фурье $ u(\rho,\theta,\phi)=\sum\limits _{n=0}^{+\infty}\sum\limits _{k=-n}^{n}R_{n}^{k}(\rho)Y_{n}^{k}(\theta,\phi) $, получаем: \begin{equation*}
Y_{n}^{k}(\theta,\phi)=\sqrt{\frac{(2n+1)(n-k)!}{4\pi(n+k)!}}P_{n}^{k}(\cos\phi)e^{ik\theta},
\end{equation*}
где $ P_{n}^{k}(x) $ --- \textit{присоединенные полиномы Лежандра}, известные своей ортогональность в $ L_{2} $: \begin{equation*}
P_{n}^{k}(x)=\frac{(-1)^{n}}{n!2^{n}}(1-x^{2})^{k/2}\frac{d^{n+k}}{dx^{n+k}}[(1-x^{2})^{n}]
\end{equation*}
Воспользуемся разложением ядра Фурье через $ Y_{n}^{k} $: \begin{equation*}
e^{i(\vec{w},\vec{r})}=4\pi\sum\limits _{n=0}^{+\infty}\sum\limits _{k=-n}^{n}(-1)^{n}\sqrt{\frac{\pi}{2r\rho}}J_{n+1/2}(\rho r)\overline{Y_{n}^{k}(\phi,\psi)}Y_{n}^{k}(\theta,\eta)
\end{equation*}
Подставляя полученное выражение в теорему о свертке:
\begin{multline*}
[f***g]=F^{-1}[ F[f]\cdot F[g] ]= \int\limits_{\mathbb{R}^3} \left[ \left( \int\limits_{\mathbb{R}^3} f(r) e^{i (\vec{w}, \vec{r})} r^2 \sin \phi dr d\phi \right) \cdot \right. \\
\left. \cdot \left( \int\limits_{\mathbb{R}^3} g(r) e^{i (\vec{w}, \vec{r})} r^2 \sin \phi dr d\phi \right) \right] e^{-i (\vec{w}, \vec{r})} \rho^2 \sin \theta d\rho d\theta d\eta
\end{multline*}
и меняя порядок суммирования с учетом ортогональности полиномов Лежандра, получаем: 
\begin{equation*}
[f***g](\vec{r})=4\pi[(r\cdot f)*g](r)
\end{equation*}