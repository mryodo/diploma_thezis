\chapter{Постановка задачи}

После введение модели стационарных сообществ Ульфа Дикмана сформулируем условие поставленной перед нами задачи:


\textbf{Задача.} Найти стационарную точку системы:
\begin{equation*}
\forall i,j=1,2:\qquad\begin{cases}
\frac{dN_{i}(t)}{dt}=0,\\
\;\;\;\vdots\\
\frac{dC_{ij}(\xi,t)}{dt}=0
\end{cases}
\end{equation*}
при условии, что 
\begin{itemize}
	\item $ i,j=1,\;2 \forall i,j $ функция $ C_{ij}(\xi,t) $ радиально-симметрична; 
	\item ядра рождаемости $ m_{i}(\xi) $ и конкуренции $ w_{ij}(\xi) $ распределены нормально с нулевым математическим ожиданием;
	\item для разрешения иерархии зависимостей использовать замыкание \ref{eq:closure} c $\alpha=2/5$ \begin{multline*}
T_{ijk}(\xi,\xi')=\frac{1}{5}\left(\frac{C_{ij}(\xi)C_{ik}(\xi')}{N_{i}}+\frac{C_{ij}(\xi)C_{jk}(\xi'-\xi)}{N_{j}}+\right.\\
\left.+\frac{C_{ik}(\xi')C_{jk}(\xi'-\xi)}{N_{k}}-1\right)+\frac{3}{5}\cdot\frac{C_{ij}(\xi)C_{ik}(\xi')}{N_{i}};
\end{multline*}
\end{itemize}
С учетом того, что динамика системы описывается уравнениями:
\begin{align*}
\frac{d}{dt}N_{i}=(b_{i}-d_{i})N_{i}-\sum_{j}\int_{\mathbb{R}^{n}}w_{ij}(\xi)C_{ij}(\xi)d\xi \\
\frac{d}{dt}C_{ij}(\xi)	=	\delta_{ij}m_{i}(-\xi)N_{i}+\int_{\mathbb{\mathbb{R}}^{n}}m_{i}(\xi')C_{ij}(\xi+\xi')d\xi'-d_{i}C_{ij}(\xi)-\\
-\sum_{k}\int_{\mathbb{\mathbb{R}}^{n}}w_{ik}(\xi')T_{ijk}(\xi,\xi')d\xi-w_{ij}(\xi)C_{ij}(\xi)+\langle i,j,\xi\to j,i,-\xi\rangle
\end{align*}

\section{Результирующая система}
Прежде чем обратиться к выведению получающейся системы, введем несколько дополнительных обозначений и договоренностей:
\begin{itemize}
	\item поскольку изучается стационарное состояние системы, будем опускать зависимость функций-пространственных моментов от времени для удобства записи;
	
	\item для приведения системы уравнений к виду, удобному для решения численным методом переобозначим искомые функции следующим образом, нормируя и центрируя на пределы:
	\begin{equation*}
	C_{ij}(\xi)=\frac{C_{ij}(\xi)}{N_iN_j},
	\qquad
	 T_{ijk}(\xi, \xi')=\frac{T_{ijk}(\xi, \xi')}{N_iN_jN_k} 
	\end{equation*}
	\begin{equation*}
	D_{ij}(\xi)=C_{ij}(\xi)-1
	\end{equation*}
	Таким образом, $\lim\limits_{\xi \to \infty} C_{ij}(\xi)=1$, $\lim\limits_{\xi \to \infty} D_{ij}(\xi)=0$;
	\item свертку двух функций будем обозначать $\int_{\mathbb{R}^n} f(y)g(x+y)dx=[f*g](x)$;
	\item введем упрощающее обозначение $y_{ij}=\int\limits_{\mathbb{R}^n} w_{ij}(\xi)C_{ij}(\xi)d\xi$ --- агрегированный конкурентный фон с учетом пространственной структуры.
\end{itemize}

Пользуясь радиальной симметричностью функций-ядер и тем фактом, что $C_{21}(\xi)=C_{12}(-\xi)$, получаем:
\begin{equation*}
[m_{2}*C_{21}](-\xi)=\int_{-\infty}^{+\infty}m_{2}(\xi')C_{21}(\xi'-\xi)d\xi'=\int_{-\infty}^{+\infty}m_{2}(\xi')C_{12}(\xi-\xi')=[m_{2}*C_{12}](\xi)
\end{equation*}
Запишем уравнение на $C_{12}(\xi)$, раскрывая знаки суммирования:
\begin{multline*}
\frac{d}{dt}C_{12}(\xi)=[m_{1}*C_{12}](\xi)-d_{1}C_{12}(\xi)-\\
-\left(d{}_{11}^{'}\int w_{11}(\xi')T_{121}(\xi,\xi')d\xi'+d'_{12}\int w_{22}(\xi')T_{122}(\xi,\xi')d\xi'\right)-\\
-w_{12}(\xi)C_{12}(\xi)+[m_{2}*C_{21}](-\xi)-d_{2}C_{21}(-\xi)-\\
-\left(d_{21}^{'}\int w_{21}(\xi')T_{211}(-\xi,\xi')d\xi'+d'_{22}\int w_{22}(\xi')T_{212}(-\xi,\xi')d\xi'\right)-w_{21}(-\xi)C_{21}(-\xi)
\end{multline*}
Применим полученное отношение на свертки функций с разным знаком:
\begin{multline*}
\frac{d}{dt}C_{12}(\xi)=-\left(\int w_{21}(\xi')T_{211}(-\xi,\xi')d\xi'+\int w_{22}(\xi')T_{212}(-\xi,\xi')d\xi'+\right.\\
\left.+\int w_{11}(\xi')T_{121}(\xi,\xi')d\xi'+\int w_{12}(\xi')T_{122}(\xi,\xi')d\xi'\right) + \\
+[(m_{1}+m_{2})*C_{12}](\xi)-(d_{1}+d_{2}+w_{12}(\xi)+w_{21}(\xi))C_{12}(\xi)
\end{multline*}
В нормированном виде 
\begin{multline*}
\frac{d}{dt}C_{12}(\xi)=[(m_{1}+m_{2})*C_{12}](\xi)-(d_{1}+d_{2}+w_{12}(\xi)+w_{21}(\xi))C_{12}(\xi)-\\
-(N_{1}\int w_{21}(\xi')T_{211}(-\xi,\xi')d\xi'+N_{2}\int w_{22}(\xi')T_{212}(-\xi,\xi')d\xi'+\\
+N_{1}\int w_{11}(\xi')T_{121}(\xi,\xi')d\xi'+N_{2}\int w_{12}(\xi')T_{122}(\xi,\xi')d\xi')
\end{multline*}
Необходимо подставить нормированное замыкание, которые принимает слеюущий вид:
\begin{multline*}
T_{ijk}(\xi,\xi')=\frac{\alpha}{2}(C_{ij}(\xi)C_{ik}(\xi')+C_{ij}(\xi)C_{jk}(\xi-\xi')+C_{ik}(\xi')C_{jk}(\xi-\xi')-1)+(1-\alpha)C_{ij}(\xi)C_{ik}(\xi')
\end{multline*}
В итоге получаем (для удобства чтения, в каждой строке сгруппированы слагаемые с одним и тем же параметрическим множителем, который указан в правом нижем углу):
\begin{align*}
N_{1}\int w_{21}(\xi')T_{211}(-\xi,\xi')d\xi'+N_{2}\int w_{22}(\xi')T_{212}(-\xi,\xi')d\xi'+		\\
+N_{1}\int w_{11}(\xi')T_{121}(\xi,\xi')d\xi'+N_{2}\int w_{12}(\xi')T_{122}(\xi,\xi')d\xi'	|_{C(\xi)C(\xi')}=	\\
=N_{1}C_{12}y_{21}+N_{2}C_{12}y_{22}+N_{1}C_{12}y_{11}+N_{2}C_{12}y_{12} &&		\times(1-\frac{\alpha}{2})
\end{align*}
где подстановка $T_{ijk}(\xi, \xi')=C_{ij}(\xi)C_{ik}(\xi')$ обозначена как 	$|_{C(\xi)C(\xi')}$.

Воспользуемся уравнениями на первые моменты:
\begin{equation*}
\begin{cases}
N_{1}y_{11}+N_{2}y_{12}=b_{1}-d_{1}\\
N_{1}y_{21}+N_{2}y_{22}=b_{2}-d_{2}
\end{cases}
\end{equation*}
Получаем:
\begin{align*}
N_{1}\int w_{21}(\xi')T_{211}(-\xi,\xi')d\xi'+N_{2}\int w_{22}(\xi')T_{212}(-\xi,\xi')d\xi'+		\\
+N_{1}\int w_{11}(\xi')T_{121}(\xi,\xi')d\xi'+N_{2}\int w_{12}(\xi')T_{122}(\xi,\xi')d\xi'	|_{C(\xi)C(\xi')}=	\\
=(b_{1}+b_{2}-d_{1}-d_{2})C_{12}(\xi)	=	\\
=(b_{1}+b_{2}-d_{1}-d_{2})D_{12}(\xi)+(b_{1}+b_{2}-d_{1}-d_{2}) &&		\times(1-\frac{\alpha}{2})
\end{align*}
По линейности интеграла, подставим замыкание и раскроем получающееся выражение:
\begin{align*}
N_{1}\int w_{11}(\xi')T_{121}(\xi,\xi')d\xi'+N_{2}\int w_{12}(\xi')T_{122}(\xi,\xi')d\xi'+		\\
+N_{1}\int w_{21}(\xi')T_{211}(-\xi,\xi')d\xi'+N_{2}\int w_{22}(\xi')T_{212}(-\xi,\xi')d\xi'|_{C(\xi)C(\xi-\xi')}	=	\\
=N_{1}C_{12}[w_{11}*C_{21}]+N_{2}C_{12}[w_{12}*C_{22}]+N_{1}C_{12}[w_{21}*C_{11}]+N_{2}C_{12}[w_{22}*C_{12}]	= \\	
=(D_{12}+1)(N_{1}[w_{11}*D_{21}]+N_{2}[w_{12}*D_{22}]+N_{1}[w_{21}*D_{11}]+N_{2}[w_{22}*D_{12}]+ \\		
+N_{2}d'_{22}+N_{1}d'_{21}+N_{2}d_{22}^{'}+N_{1}d{}_{11}^{'})		&& \times\frac{\alpha}{2}
\end{align*}
Аналогично поступим для оставшегося блока, также воспользовавшись уравнениями для динамики первых моментов:
\begin{align*}
N_{1}\int w_{21}(\xi')T_{211}(-\xi,\xi')d\xi'+N_{2}\int w_{22}(\xi')T_{212}(-\xi,\xi')d\xi'		\\
+N_{1}\int w_{11}(\xi')T_{121}(\xi,\xi')d\xi'+N_{2}\int w_{12}(\xi')T_{122}(\xi,\xi')d\xi'|_{C(\xi')C(\xi-\xi')}	=	\\
=N_{1}[w_{21}C_{12}*C_{11}]+N_{2}[w_{22}C_{22}*C_{12}]+N_{1}[w_{11}C_{11}*C_{12}]+N_{2}[w_{12}C_{12}*C_{22}]	=	\\
=\{N_{1}[(w_{12}D_{12}+w_{12})*(D_{22}+1)]\}	=	\\
=N_{1}[w_{21}D_{12}*D_{11}]+N_{1}[w_{21}*D_{11}]+N_{1}y_{21}	+	\\
+N_{2}[w_{12}D_{12}*D_{22}]+N_{2}[w_{12}*D_{22}]+N_{2}y_{12}	+	\\
+N_{1}[w_{11}D_{11}*D_{12}]+N_{1}[w_{11}*D_{12}]+N_{1}y_{11}	+	\\
+N_{2}[w_{22}D_{22}*D_{12}]+N_{2}[w_{22}*D_{12}]+N_{2}y_{22}	=	\\
=N_{1}[w_{21}*D_{11}]+N_{2}[w_{22}*D_{12}]+N_{1}[w_{11}*D_{12}]+N_{2}[w_{12}*D_{22}]	+	\\
+N_{1}[w_{21}D_{12}*D_{11}]+N_{2}[w_{22}D_{22}*D_{12}]+N_{1}[w_{11}D_{11}*D_{12}]+N_{2}[w_{12}D_{12}*D_{22}]	+	\\
+N_{1}d'_{21}+N_{2}d'_{22}+N_{1}d'_{11}+N_{2}d'_{12}	+	\\
+b_{1}+b_{2}-d_{1}-d_{2}	&&	\times\frac{\alpha}{2}
\end{align*}
Проссумируем три получившихся блока:
\begin{align*}
((1-\frac{\alpha}{2})(b_{1}+b_{2})+\frac{\alpha}{2}(d_{1}+d_{2})+w_{12}+w_{21})D_{12}=[(m_{1}+m_{2})*D_{12}]-w_{12}-w_{21}		\\
-\frac{\alpha}{2}(D_{12}(N_{1}[w_{11}*D_{21}]+N_{2}[w_{12}*D_{22}]+N_{1}[w_{21}*D_{11}]+		\\
+N_{1}[w_{21}D_{12}*D_{11}]+N_{2}[w_{22}D_{22}*D_{12}]+N_{1}[w_{11}D_{11}*D_{12}]+N_{2}[w_{12}D_{12}*D_{22}]) + \\
+2N_{1}[w_{11}*D_{21}]+2N_{2}[w_{12}*D_{22}]+2N_{1}[w_{21}*D_{11}]+2N_{2}[w_{22}*D_{12}]+	\\			
N_{2}[w_{22}*D_{12}]+N_{2}d'_{12}+N_{1}d'_{21}+N_{2}d_{22}^{'}+N_{1}d{}_{11}^{'})
\end{align*}
И приведем к итоговому виду:
\begin{align*}
((1-\frac{\alpha}{2})(b_{1}+b_{2})+\frac{\alpha}{2}(d_{1}+d_{2}+d'_{11}N_{1}+d'_{12}N_{2}+d'_{21}N_{1}+d'_{22}N_{2})+w_{12}+w_{21})D_{12}=	\\
[(m_{1}+m_{2})*D_{12}]-w_{12}-w_{21}-\\
-\frac{\alpha}{2}N_{1}((D_{12}+2)([w_{11}*D_{12}]+[w_{21}*D_{11}])+[w_{21}D_{12}*D_{11}]+[w_{11}D_{11}*D_{12}])-\\		
-\frac{\alpha}{2}N_{2}((D_{12}+2)([w_{12}*D_{22}]+[w_{22}*D_{12}])+[w_{22}D_{22}*D_{12}]+[w_{12}D_{12}*D_{22}])		
\end{align*}
Аналогичным образом преобразуем два оставшихся уравнения:
\begin{align*}
((1-\frac{\alpha}{2})b_{1}+\frac{\alpha}{2}(d_{1}+N_{1}d'_{11}+N_{2}d'_{12})+w_{11})D_{11}=\frac{m_{1}}{N_{1}}+\left[m_{1}*D_{11}\right]-w_{11}-\\
-\frac{\alpha}{2}N_{1}((D_{11}+2)[w_{11}*D_{11}]+[w_{11}D_{11}*D_{11}])-\\
-\frac{\alpha}{2}N_{2}((D_{11}+2)[w_{12}*D_{12}]+[w_{12}D_{12}*D_{12}])
\\
((1-\frac{\alpha}{2})b_{2}+\frac{\alpha}{2}(d_{2}+N_{1}d'_{21}+N_{2}d'_{22})+w_{22})D_{22}=\frac{m_{2}}{N_{2}}+\left[m_{2}*D_{22}\right]-w_{22}-\\
-\frac{\alpha}{2}N_{2}((D_{22}+2)[w_{22}*D_{22}]+[w_{22}D_{22}*D_{22}])-\\
-\frac{\alpha}{2}N_{1}((D_{22}+2)[w_{21}*D_{12}]+[w_{21}D_{12}*D_{12}])
\end{align*}
И решим линейную систему на $ N_{1} $ и $ N_{2} $:
\begin{align*}
N_{1}=\frac{(b_{1}-d_{1})y_{22}-(b_{2}-d_{2})y_{12}}{y_{11}y_{22}-y_{12}y_{21}}
\\
N_{2}=\frac{(b_{2}-d_{2})y_{11}-(b_{1}-d_{1})y_{21}}{y_{11}y_{22}-y_{12}y_{21}}
\end{align*}
Получившаяся система из пяти нелинейных интегральных уравнений описывает стационарное положение рассматриваемой популяции в пространстве.