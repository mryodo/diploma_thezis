\chapter{Модель самоструктурирующихся сообществ Ульфа Дикмана}

В данной части работы будет обсуждаться модель, учитывающая пространственную структуру сообщества, предложенную в [1, 2] в случае двухвидовых популяций. Здесь мы намеренно не станем затрагивать вопрос о формализации некоторых переходов и величин, используемых нами, поскольку это было проведено в работе [22] и, как было доказано, не сказывается на итоговых уравнениях системы.

\section{Пространственные моменты}

Наша модель будет рассматривать события, происходящие с каждым индивидом, находящимся в конкретной точке пространства. Договоримся считать, что вся популяция у нас ограничена на конечной области пространства, обозначаемой $ A $. Длина, площадь или объем этой области в зависимости от ее размерности будем обозначать $ |A| $.

\textbf{Определение 1.} Распределение особей в области $ A $ в конкретный момент будем называть паттерном $ p(x) $. Пусть в популяции участвует n различных видов. Тогда  \begin{equation*}
p(x)=\left(p_{1}(x),\;p_{2}(x),\ldots,\:p_{n}(x)\right), 
\end{equation*}
где $ p_{i}(x) $ --- паттерн $ i- $го вида, равный
\begin{equation*}
p_{i}(x)=\sum_{x_{o}\in X}\delta(x-x_{o}),
\end{equation*}
где $ \delta(x) $ --- дельта-функция Дирака, а $ X $ --- множество точек, в которых присутствует особь $ i- $го вида. Здесь и далее зависимость величины от $ p(x) $ предлагается читать как фразу «в конкретный момент».

Такое введение пространственного паттерна позволяет довольно просто выразить среднюю плотность индивидов $ i- $го вида в области:
\begin{equation*}
N_{i}(p)=\frac{1}{|A|}\int_{A}p_{i}(x)dx
\end{equation*}

\textbf{Определение 2.} Первым моментом (плотностью индивидов) $ i- $го вида называется математическое ожидание средних плотностей $ N_{i}(p) $ по всему пространству паттернов:
\begin{equation*}
N_{i}(t)=\mathbb{E}_{p}N_{i}(p)
\end{equation*}

Как уже говорилось выше, вопрос корректности, меры в данном пространстве и правомерности последующей работы с данной величиной доказан в [22].

Аналогично можно ввести корреляционную функцию, то есть число пар индивидов определенных видов $ i $ и $ j $, находящихся на расстоянии $ \xi $:
\begin{equation*}
C_{ij}(\xi,p)=\frac{1}{|A|}\int_{A}p_{i}(x)[p_{j}(x+\xi)-\delta_{ij}\delta_{x}(x+\xi)]dx
\end{equation*}

\textbf{Определение 3.} Вторым пространственным моментов (плотностью пар) особей видов $ i $ и $ j $ таких, что особь вида $ i $ находится на расстоянии $ \xi $ от особи вида $ j $, будем называть
\begin{equation*}
C_{ij}(\xi,t)=\mathbb{E}_{p}C_{ij}(\xi,p)
\end{equation*}

\textbf{Замечание 4.} Важно отметить следующий момент: в рамках нашей модели будем полагать, что взаимодействия между индивидами наблюдаются только на сравнительно малых расстояниях (явление, широкоизвестное как plant's eye view). Поэтому на достаточно больших расстояниях можно считать, что пространственная структура отсутствует, т.е.
\begin{equation*}
\lim_{\xi\to\infty}C_{ij}(\xi,t)=N_{i}(t)N_{j}(t)
\end{equation*}

Аналогично, можно определить плотность и более общих пространственных структур: 
\begin{equation*}
C_{i_{1}\ldots i_{m}}(\xi_{1},\ldots,\xi_{m-1},p)=\frac{1}{|A|}\int_{A}p_{i}(x)\prod_{j=2}^{m}p_{j_{j}}(x+\xi_{j-1})dx
\end{equation*}
\begin{equation*}
C_{i_{1}\ldots i_{m}}(\xi_{1},\ldots,\xi_{m-1},t)=\mathbb{E}_{p}C_{i_{1}\ldots i_{m}}(\xi_{1},\ldots,\xi_{m-1},p)\text{,}
\end{equation*}
договорившись, что $ \xi_{j} $ есть расстояние (в многомерном случае вектор) между индивидом вида $ i_{1} $ и $ i_{j+1} $.

В нашей работе ограничимся только введением моментов третьего порядка $ T_{ijk}(\xi,\xi',t) $ --- плотности троек индивидов.

\textbf{Замечание 5.} Как указано выше, после взятия матожидания возникает зависимость от времени. Стоит понимать, что от времени зависит вероятность реализации паттерна (так, например, равновесные паттерны более вероятны через длительное время).


\section{События динамики модели}

Будем рассматривать 3 вида событий, который могут произойти с индивидом в рамках нашей модели, --- рождение, гибель и перемещение.

\textbf{Замечание 6.} В настоящей работе положим, что все движение индивидов реализуется за счет рождения. Это может быть интерпретировано как изучение стационарных сообществ, например, сообществ растений; технически же введение движение означает корректировку одной конкретной функции, что существенно не влияет на дальнейшую задачу, но существенно осложняет формулировку механизмов сосуществования, что является конечной целью данной работы.

\subsection{Рождение нового индивида}

Вероятность рождения потомка вида i в точке $ \xi' $ от родителя, находящегося в точке $ \xi $ будем обозначать
\begin{equation*}
B_{i}(\xi,\xi')=m_{i}(\xi'-\xi),
\end{equation*}
где функцию $ m_{i}(x) $ будем называть ядром рождения (dispersal kernel).

$ \int_{\mathbb{R}^{n}}m_{i}(x)dx=b_{i} $ --- темп рождаемости, $ 0<b_{i}<1 $. В данной работе будем считать, что $ \frac{1}{b_{i}}m_{i}(x) $ распределено нормально с нулевым матожиданием $ \left(m_{i}\sim N(0,\sigma_{i}^{m})\right) $; также положим $ m_{i} $ радиально-симметричной ($ m_{i}(x)=m_{i}(|x|) $) из биологических соображений.

Заметим, что введенные ядра рождения --- это первые функции, которые несут на себе пространственную структуру.

\subsection{Гибель индивида}

Вероятность смерти конкретного индивида $ i- $го вида, находящегося в точке $ \xi $, очевидно зависит от того, как расположены оставшиеся особи в сообществе и как сильно они конкурируют с данным индивидом. Данную вероятность будем обозначать
\begin{equation*}
D_{i}(\xi,p)=d_{i}+\sum_{j}\int_{\mathbb{R}^{n}}w_{ij}\left(\xi-\xi'\right)\left[p_{j}(\xi')-\delta_{ij}\delta_{x}(\xi')\right]d\xi',
\end{equation*}
где $ d_{i} $ --- вероятность сметри от влияния окружающей среды (полагаем его пространственно постоянным), $ 0<d_{i}<1 $; $ w_{ij}(x) $ --- ядро конкуренции --- плотность вероятности сметри индивида $ i- $го вида от конкуренции с индивидом $ j- $го вида, находящимся на расстоянии $ x $, $ \int_{\mathbb{R}^{n}}w_{ij}(x)dx=d'_{ij} $ --- сила конкуренции, $ 0<d'_{ij}<1 $. В данной работе будем считать, что $ \frac{1}{d'_{ij}}w_{ij}(x) $ распределено нормально с нулевым матожиданием $ \left(w_{ij}\sim N(0,\sigma_{ij}^{w})\right) $; также положим $ w_{ij} $ радиально-симметричной ($ w_{ij}(x)=w_{ij}(|x|) $) из биологических соображений.

\section{Динамика моментов}

Теперь несложно установить, как связаны описанные выше события рождения и динамика моментов: чтобы перейти из одного паттерна в другой, нужно, чтобы произошел некоторый набор описанных событий. Устремляя время на данный переход к 0, получим производную плотности вероятности того, что мы перейдем из одного паттерна в другой, а поскольку время сколь угодно мало, то путем ровно одного события «рождение-гибель».

Чтобы не повторять технический вывод, проведенный в [2], выпишем сразу результат для первых двух моментов:
\begin{equation*}
\frac{d}{dt}N_{i}=(b_{i}-d_{i})N_{i}-\sum_{j}\int_{\mathbb{R}^{n}}w_{ij}(\xi)C_{ij}(\xi)d\xi
\end{equation*}
Несложно заметить, что все слагаемые в правой части имеют простую биологическую интерпретацию:
\begin{itemize}
\item $ b_{i}N_{i} $ --- суммарная вероятность рождения нового индивида от всех родителей;

\item $ d_{i}N_{i} $ --- суммарная вероятность смерти индивида под влиянием окружающей среды;

\item $ \sum_{j}\int_{\mathbb{R}^{n}}w_{ij}(\xi)C_{ij}(\xi)d\xi $ --- для каждого вида $ j $ суммарная вероятность смерти от конкуренции с учетом того, как часто встречаются пары вида $ i-j $.
\end{itemize}
Аналогично выпишем динамику второго момента:
\begin{multline*}
\frac{d}{dt}C_{ij}(\xi)	=	\delta_{ij}m_{i}(-\xi)N_{i}+\int_{\mathbb{\mathbb{R}}^{n}}m_{i}(\xi')C_{ij}(\xi+\xi')d\xi'-d_{i}C_{ij}(\xi)-\\
-\sum_{k}\int_{\mathbb{\mathbb{R}}^{n}}w_{ik}(\xi')T_{ijk}(\xi,\xi')d\xi-w_{ij}(\xi)C_{ij}(\xi)+\langle i,j,\xi\to j,i,-\xi\rangle
\end{multline*}

Здесь так же можно охарактеризовать каждое слагаемое:
\begin{itemize}
	\item $ \delta_{ij}m_{i}(-\xi)N_{i} $ --- в случае пары одного вида, новая такая же пара может появиться, если условно «первая» особь в паре создаст потомка на расстоянии $ -\xi $ (тогда новую пару образует потомок и родитель);
	
	\item  $ \int_{\mathbb{\mathbb{R}}^{n}}m_{i}(\xi')C_{ij}(\xi+\xi')d\xi' $ --- также новая пара может возникнуть, если «первая» особь в паре создаст потомка на расстоянии $ \xi $ от любой особи вида $ j $ (тогда новую пару создадут потомок и особь вида $ j $);
	
	\item $ d_{i}C_{ij}(\xi) $ --- пара может исчезнуть, если погибнет «первая» особь в паре из-за воздействия среды;
	
	\item $ w_{ij}(\xi)C_{ij}(\xi) $ --- пара может исчезнуть, если погибнет «первая» особь из-за конкуренции внутри пары;
	
	\item  $ \sum_{k}\int_{\mathbb{\mathbb{R}}^{n}}w_{ik}(\xi')T_{ijk}(\xi,\xi')d\xi-w_{ij}(\xi)C_{ij}(\xi) $ --- или же пара может исчезнуть, если погибнет «первая» особь в паре из-за конкуренции со всеми оставшимися особями в сообществе с учетом пространственной структуры (структуры троек, $ T_{ijk} $);
	
	\item слагаемое $ \langle i,j,\xi\to j,i,-\xi\rangle $ означает, что далее следует повторить рассуждения, но теперь уже для «второй» особи из пары с поправкой на изменение ориентации расстояния.
\end{itemize}

\section{Замыкания пространственных моментов}

Как можно заметить из вышеизложенных уравнений динамики, динамика момента $ i- $го порядка зависит от момента $ (i+1)- $го порядка.

\textbf{Замечание 7.} Динамика момента порядка $ m $ всегда зависит от момента старшего порядка, поскольку включает в себя пространственно-неоднородную конкурентную смерть, т.е. перебор всех возможных структур из $ \left(m+1\right) $ особи. 

Таким образом уравнения динамики пространственных моментов порождают иерархию зависимостей, приводящую к счетной системе интегро-дифференциальных уравнений. Для разрешения подобной иерархии зависимостей предлагается использовать классическую (в математической физике и биологии) идею замыканий.

\textbf{Определение 8.} Замыканием пространственных моментов называется выражение момента порядка m через моменты непревосходящего порядка. По сути своей, замыкание есть аппроксимация количества более сложных пространственных структур через более простые.

\textbf{Предложение 9.} \textit{(Дикман)} В [1, 2] было предложено использовать замыкания третьих моментов через моменты первого и второго порядка, $ T_{ijk}=F(C,N) $. Утверждается, что подобная схема замыканий достаточна для получения пространственно неоднородной популяции и не пренебрегает значимыми эффектами, которые могли бы появиться при рассмотрении моментов более высоких порядков.

Несложно заметить, что с аналитической точки зрения разумно замыкать моменты как можно меньшего порядка, однако здесь есть опасность потери пространственной структуры сообщества. Например, замыкания вида
\begin{equation*}
C_{ij}(\xi,t)=N_{i}(t)N_{j}(t)
\end{equation*}
\begin{equation*}
C_{ij}(\xi,t)=N_{i}(t)N_{j}(t)(1+\varphi(\xi))
\end{equation*}
Приводят к обобщенной модели Лотки-Вольтерра, которая, как уже обсуждалось во вступлении к данной работе, имеет тенденцию к потери пространственных эффектов.

\subsection{Необходимые требования на замыкания}

В то же время ясно, что не любую функцию можно считать замыканием. Более подробно процесс выбора кандидатов для замыканий описан в [4]; здесь мы коснемся основных утверждений, позволяющих сформировать некоторые необходимые требования к замыканию.
\begin{enumerate}
	\item $  T_{ijk}(\xi,\xi')\ge0 $;
	
	\item $ T_{ijk}(\xi,\xi')=T_{jik}(-\xi,\xi'-\xi)=T_{kij}(-\xi',\xi-\xi') $ --- фактически, это правило треугольника, в котором мы переставляем вершины;
	
	\item если$  C_{ij}=N_{i}N_{j} $, то $ T_{ijk}=N_{i}N_{j}N_{k} $, т.е. если отсутствует пространственная структура на уровне пар, то сообщество так же однородно с точки зрения троек;
	
	\item $ \lim_{\xi\to\infty}T_{ijk}(\xi,\xi')=N_{i}C_{jk}(\xi'-\xi) $ --- удаленная точка уничтожает пространственную структуру у двух сторон треугольника из трех;
	
	\item аналогично $ \lim_{\xi'\to\infty}T_{ijk}(\xi,\xi')=C_{ij}(\xi)N_{k} $
	
	\item $  \frac{1}{|A|}\int_{\mathbb{R}^{n}}T_{ijk}(\xi,\xi')d\xi'=C_{ij}(\xi)N_{k} $ --- суммарное количество троек есть количество пар на плотность оставшегося вида.
\end{enumerate}

Как видно, все эти требования исходят из того, что в случае, если исчезнет пространственная структура в следствие самоструктурирования сообщества или увеличения расстояния, замыкание должно продолжать оставаться верным.

\subsection{Достаточное требование на замыкания}

В то время как удается найти довольно много различных необходимых требований на замыкания, конечный выбор используемого определяется ровно одним крайне не аналитическим утверждением:

\textbf{Определение 10.} Замыкание считается корректным, если оно удовлетворяет всем необходимым требованиям и позволяет хорошо приблизить результаты компьютерных симуляций.

С результатами вычислений в [2] сравнивалось несколько кандидатов:
\begin{equation*}
T_{ijk}(\xi,\xi')\approx C_{ij}(\xi)N_{k}+C_{ik}(\xi')N_{j}+C_{jk}(\xi-\xi')N_{i}-2N_{i}N_{j}N_{k};
\end{equation*}
\begin{equation*}
T_{ijk}(\xi,\xi')\approx\frac{1}{2}\left[\frac{C_{ij}(\xi)C_{ik}(\xi')}{N_{i}}+\frac{C_{ij}(\xi)C_{jk}(\xi'-\xi)}{N_{j}}+\frac{C_{ik}(\xi')C_{jk}(\xi'-\xi)}{N_{k}}-N_{i}N_{j}N_{k}\right];
\end{equation*}
\begin{equation*}
T_{ijk}(\xi,\xi')\approx\frac{C_{ij}(\xi)C_{ik}(\xi')}{N_{i}};
\end{equation*}
\begin{equation*}
T_{ijk}(\xi,\xi')\approx\frac{C_{ij}(\xi)C_{ik}(\xi')C_{jk}(\xi'-\xi)}{N_{i}N_{j}N_{k}}.
\end{equation*}
На рисунке [fig:compclos] приводятся результат сравнения кандидатов в замыкания с компьютерными симуляциями на фазовых портретах в пространстве $ \left[N_{1};N_{2}\right] $. Как несложно заметить, наиболее точно под симуляции подходит замыкание [eq:cl3], названное асимметричным или независимым (по аналогии с вероятностью независимых величин).

Однако как было показано в [23], использование данного замыкания в случае одного вида приводит к необходимому $ d_{1}\ne0 $, то есть отсутствию влияния внешней среды. Данный результат был использован в работе [24], где было получено, что в случае двухвидовой популяции использование замыкания [eq:cl3] влечет $ d_{1}=d_{2}=d'_{11}=d'_{22}=d'_{12}=d'_{21}=0 $, что биологически несостоятельно, поскольку требует полного отсутствия межвидовой и внутривидовой конкуренции.

Также в [24] было предложено использовать замыкание [eq:cl2] для третьих моментов, замкнутых на отдельный вид, т.е. $ T_{iii}(\xi,\xi') $, что привело к разрешимой системе, однако в ней не обнаружилось предсказанных симуляциями пространственных эффектов, т.е. подобная схема замыканий не удовлетворяла достаточному требованию.

В [3] было исследовано параметрическое семейство замыканий:
\begin{multline}\label{eq:closure}
T_{ijk}(\xi,\xi')	=	\frac{\alpha}{2}\left(\frac{C_{ij}(\xi)C_{ik}(\xi')}{N_{i}}+\frac{C_{ij}(\xi)C_{jk}(\xi'-\xi)}{N_{j}}+\right.\\
\left.+\frac{C_{ik}(\xi')C_{jk}(\xi'-\xi)}{N_{k}}-1\right)+(1-\alpha)\frac{C_{ij}(\xi)C_{ik}(\xi')}{N_{i}},
\end{multline}
которое удовлетворяет всем необходимым требованиям и является комбинацией замыканий [eq:cl2] и [eq:cl3]. Как видно из рисунка [fig:compclos-1], лучше всего под симуляции подходит параметрическое замыкание с $ \alpha=\frac{4}{5} $. В то же время по своей структуре оно во многом схоже с замыканием [eq:cl2], а значит, как следует из [24], приведет к численно разрешимой системе нелинейных интегральных уравнений. Основываясь на данных результатах для работы было выбрано данное замыкание.