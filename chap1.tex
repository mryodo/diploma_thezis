\chapter{Введение}
\label{chap:Intro}

%% Restart the numbering to make sure that this is definitely page #1!
\pagenumbering{arabic}

%% Note that the citations in this chapter use the journal and
%% arXiv keys: I used the SLAC-SPIRES online BibTeX retriever
%% to build my bibliography. There are also quite a few non-standard
%% macros, which come from my personal collection. You can have them
%% if you want, or I might get round to properly releasing them at
%% some point myself.

%\chapterquote{Laws were made to be broken.}%
%{Christopher North, 1785--1854}%: Blackwood's Magazine May 1830

% Антон Савостьянов, [24.01.17 15:31]
%Ячейки бенара

%Антон Савостьянов, [24.01.17 15:32]
%Лизеганга

%Антон Савостьянов, [24.01.17 15:34]
%Белоусова Жаботинского

%Антон Савостьянов, [24.01.17 15:41]
%Dictostelium discoideum

%Антон Савостьянов, [24.01.17 15:51]
%E. Coli

В 1916 году в \cite{einstein} Альберт Эйнштейн (Albert Einstein) предложил концепцию вынужденного излучения (stimulated emission) --- колебаний возбужденных электронов, индуцированных существующей световой волной: согласно предложенной теории, данный процесс порождает набор разнофазовых эквиамплитудных волн, \emph{конкурентная самоорганизация} которых в стацонарном положении образует равномерно колеблющуюся волну.Таким образом, процесс самоорганизации мультиагентной естественной системы лежит в основе явления, известного как \emph{лазерное излучение} \cite{steen}. 
