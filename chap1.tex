\chapter{Введение}
\label{chap:Intro}

%% Restart the numbering to make sure that this is definitely page #1!
\pagenumbering{arabic}

%% Note that the citations in this chapter use the journal and
%% arXiv keys: I used the SLAC-SPIRES online BibTeX retriever
%% to build my bibliography. There are also quite a few non-standard
%% macros, which come from my personal collection. You can have them
%% if you want, or I might get round to properly releasing them at
%% some point myself.

%\chapterquote{Laws were made to be broken.}%
%{Christopher North, 1785--1854}%: Blackwood's Magazine May 1830

% Антон Савостьянов, [24.01.17 15:31]
%Ячейки бенара

%Антон Савостьянов, [24.01.17 15:32]
%Лизеганга

%Антон Савостьянов, [24.01.17 15:34]
%Белоусова Жаботинского

%Антон Савостьянов, [24.01.17 15:41]
%Dictostelium discoideum

%Антон Савостьянов, [24.01.17 15:51]
%E. Coli

В 1916 году в \cite{einstein} Альберт Эйнштейн (Albert Einstein) предложил концепцию вынужденного излучения (stimulated emission) --- возникновения колебаний возбужденных электронов, индуцированное существующей световой волной: согласно предложенной теории, данный процесс создает набор разнофазовых эквиамплитудных волн, в стационарном положении порождающий равномерно колеблющуюся волну, также известную как \emph{лазерное излучение} \cite{steen}. Описанный эффект есть ничто иное как \emph{конкурентная самоорганизация мультиагентных процессов} в естественных системах. 

Подобные явления подробно изучались в случае химических систем: в 1896 Рафаэелем Лизегангом (Raphael E. Liesegang)  был рассмотрен процесс формирования структур, являющихся следствием выпадения в осадок вещества, получившегося в результате химической реакции, (кольца Лизеганга) \cite{leis}. Другими широко известными феноменами являются реакция Белоусова-Жаботинского \cite{belousov} и ячейки Релея-Бенара \cite{benard}. Аналогичные процессы можно наблюдать в колониях микробактерий \emph{Escherichia coli}, которые образуют слоистую пространственную структуру в целях увеличения эффективности роста популяции \cite{ecoli}, или \textit{Dictostelium discoideum}, колонии которых под воздействием внешней среды могут порождают сложные многоклеточные конструкции \cite{disc}. Более существенным практическим примером подобных явлений стоит считать процесс формирования очагов и динамику распространения заболеваний, что является классической задачей эпидемиологии; первоочередную роль здесь играет точность предсказания скорости передачи инфекции  \cite{HUPPERT2013999}. Похожим образом подобные явления возникают в неврологии при описании активности мозга и анализе поведения как возникающих отдельно стоящих структур задействованной нейронной сети, а также моделировании развития дегенеративных нейрозаболеваний, способствующих очаговому разрушению нейронных цепочек \cite{Rabinovich:2010}. Стоит отдельно отметить, что в обеих описанных выше областях аналитическая работа ведется в большинстве случаев без учета пространственно-неоднородного характера взаимодействий классическими методами логистических моделей.

Приведенные примеры пространственной самоорганизации привели к необходимости создания аналитического аппарата, т.е. набора математических моделей, априори заточенных на возможное самоструктурирование изучаемой системы. 

Необходимо отметить, что широко известные к середине \textit{XX}-го века модели конкурирующих популяций, такие как модель Ферхюльста \cite{ferh} и модель Лотки-Вольтерры \cite{Lotka1925}, исходящие из общей концепции теории самосогласованного поля (mean-field theory), не рассматривают влияние пространственный структуры сообщества, возникающей в процессе существования популяции, на динамику ее развития; в то же время, компьютерное моделирование системы как набора стохастических объектов, где каждое событие носит вероятностный характер, \cite{Huston, JUDSON19949, lom}, позволяет исследовать зависимость эволюции популяции относительно ее пространственной структуры. Полярность описанных подходов, условно, чисто аналитического \cite{ferh,Lotka1925} и чисто симуляционного, осложняется так же набором очевидных недостатков для каждого метода: в случае аналитических моделей, неспособность обрабатывать самоструктурирование сообщества как существенный фактор его развития (явления, с достаточной степенью достоверности показанные при помощи компьютерных симуляций в \cite{levin74, levin76,hols,pacsil,wein,weinc}); в случае симуляционного подхода мультиагентность рассматриваемого случайного процесса, т.е. процесса, учитывающего вероятностный характер всех событий для каждого индивида популяции, число которых меняется по времени, делает невозможной указание существенного доверительного интервала на получающийся набор статистик (плотность популяции, корреляционные функции индивидов и т.п.), вследствие чего методы следует считать малорезультативными из-за невозможности отличить значимый паттерн от случайного шума.

Ввиду описанных выше причин модели теоретической экологии прошли существенную эволюцию с середины второй половины \textit{XX} века, конечной целью которой является создание приемлемой для аналитической работы модели, результаты который не будут противоречить явлениям, описанным стохастическими симуляциями, классическим подходом к которым принято считать вероятностные клеточные автоматы \cite{HOGEWEG,velazquez}; более подробно процесс развития данных моделей описан в \cite{Plank2015}.

Одной из первых попыток моделирования самоорганизации индивидов в пространстве был набор моделей \cite{DURRETT1994363}, исходящих из предположения о том, что центральным триггером структуризации является неравномерность распределения общего ресурса; анализ таких подходов, проведенный в \cite{DURRETT1994363} показал необходимость построения полностью дискретной модели, учитывающей события, происходящие с каждым индивидом в отдельности (individual-based models). Аналогичным образом был рассмотрен подход парных корреляционных функций (pair correlation densities) \cite{YOUNG,FILIPE2001603}, как расширения общих моделей mean-field theory с образующим репродуктивным фактором. Также подробно изучался вопрос о промежуточной индивидуализации модели, при которой рассматривается конкурентное взаимодействие кластеров индивидов, образующих прото-агента популяции; конкуренция внутри кластера при этом не рассматривается \cite{Etienne2004105,metz,ROUS,CADET}.

Генеральным предположением, на котором базируются описываемые модели, является гипотеза о том, что несмотря на значительное число степеней свободы популяции, где исследуются события для каждого индивида, количество естественных степеней, влияющих на стационарное или квази-стационарное положение, можно существенно сократить \cite{RAND}; таким образом, возникает вопрос о корректном выборе пространства состояний системы, изучавшийся в \cite{WEINS1993,Hastings1994,Levin334}. Изучаемая в настоящей работе общая модель, предложенная в \cite{law_dieckmann_2000} как продолжение и формализация модели \cite{BOLKER1997179}, рассматривает пространство состояний системы в виде набора пространственных моментов, средних ожидаемых плотностей пространственных структур различных порядков; подобный подход позволяет построить естественное пространственное обобщение логистической модели \cite{ferh} как систему интегро-дифференциальных уравнений, описывающую динамику модели. Возникающая в процессе выбора пространства состояний модели счетная иерархия рассматриваемых статистик разрешается с помощью аппроксимации (замыкания) пространственных структур старшего порядка через младшие; данный подход изучался в случаях дискретного пространства  \cite{MATS,RAND1999,baalen_2000} и непрерывного пространства \cite{law_dieckmann_2000,Filipe200315}. В работе \cite{law_2003} было показано, что в случаях, когда классическое логистическое уравнение противоречит результатам компьютерных стохастических симуляций, предложенная в \cite{law_dieckmann_2000} модель работает корректно.

Концепция двухвидовых сообществ, в которых пространственные параметры взаимодействий между индивидами одного и разных видов (конкуренция и рождение) существенно влияют на стационарное положение системы, предложенная в \cite{law_2003}, была подробнее исследована в работах \cite{MURRELL,velazquez}; в частности, изучаются ограничения на пространственные и однородные параметры системы, приводящие к нетривиальным стационарным положениям системы (\textit{механизмы сосуществования}).

В настоящей работе исследуются механизмы сосуществования, предложенные в \cite{MURRELL}, при помощи модели \cite{law_dieckmann_2000} с выбранным в \cite{law_2003} замыканием пространственных моментов; главной целью исследования является изучение влияния размерности пространства на стационарные положения в случае описанных механизмов, а также усиление эффектов данных механизмов. Существенное влияние размерности пространства обитания показано в \cite{Pawar2012a} в случае тропических лесов, где при помощи анализа собранных данных было установлено, что размерность пространства сказывается не только на количественных показателях стационарных положений, но и на характере роста популяции к данному положению.

Первая часть данной дипломной работы вводит изучаемую модель биологических сообществ, описывающую динамику пространственно-неоднородной популяции, состоящей из двух конкурирующих видов; вторая часть содержит постановку задачи и вывод системы нелинейных интегральных уравнений, описывающих стационарные положения этой системы; в третьей части описан численный метод с экспоненциальной скоростью сходимости, разработанный для решения полученной системы нелинейных интегральных уравнений; в частности, приведен ряд математических эвристик, позволяющих свести вычислительную сложность двумерных и трехмерных случаев к одномерному случаю; в четвертой части работы содержатся результаты работы численного метода в случае описанных в \cite{MURRELL} механизмов сосуществования; проведен сравнительный анализ данных методов относительно размерности пространства обитания индивидов. 